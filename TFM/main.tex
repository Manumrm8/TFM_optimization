\documentclass[12pt,a4paper]{book}

% Formato del documento
\usepackage[papersize={210mm,297mm},inner=3.5cm,outer=2cm,top=2.5cm,bottom=2.5cm]{geometry}
\renewcommand{\baselinestretch}{1}
\setlength{\parskip}{8pt}

\usepackage[utf8]{inputenc}
\usepackage{amssymb}
\usepackage{amsmath}
\usepackage{graphicx}
\usepackage{tikz}
\usepackage{tkz-graph}
\usepackage{hyperref}
\usepackage[parfill]{parskip}
\usepackage{float}

\renewcommand{\chaptername}{Capítulo}
\renewcommand{\contentsname}{Índice}
\renewcommand{\bibname}{Bibliografía}
\renewcommand{\figurename}{Figura}

\definecolor{verde_oscuro}{rgb}{0.0, 0.5, 0.0}

\newtheorem{defi}{Definición}[section]
\newtheorem{prop}{Proposición}[section]
\newtheorem{propi}{Propiedades}[section]
\newtheorem{lema}{Lema}[section]
\newtheorem{tma}{Teorema}[section]
\newtheorem{cor}{Corolario}[section]
\newtheorem{nota}{Nota}[section]
\newtheorem{ejem}{Ejemplo}[section]

\hypersetup{
    colorlinks=true,
    linkcolor=blue,
    filecolor=magenta,      
    urlcolor=cyan,
    pdftitle={Overleaf Example},
    pdfpagemode=FullScreen,
    }

\urlstyle{same}

% Cabecera
\usepackage{fancyhdr}
\pagestyle{fancy}
\fancyhf{}
\renewcommand{\chaptermark}[1]{\markboth{\textbf{Cap\'itulo \thechapter}}{}} 
\fancyhead[LO]{Cap\'itulo \thechapter \quad P\'agina \thepage}
\fancyhead[RO,LE]{\textbf{\thepage}}

%%%%%%%  Comando Portada  %%%%%%%%%%%
\newcommand{\nuevaportada}[6]{
    \thispagestyle{empty}
    \begin{center}
        \includegraphics[width=0.5\textwidth]{images/logo.png}
        
        \vspace{0.5cm}
        {\Large\bfseries\textsc{M\'aster Universitario en #1}\par}
        
        \vspace{0.5cm}
        \includegraphics[width=0.4\textwidth]{images/uv.png}
        
        \vspace{0.5cm}
        {\Large\bfseries\textsc{Trabajo de Fin de M\'aster}\par}
        
        \vspace{0.5cm}
        {\Large\bfseries #2\par}
        
        \vspace{2cm}
        \begin{flushright}
            \begin{tabular}{l} 
                {\large\bfseries\textsc{Autor:}} \\
                {\large\textsc{#3}} \\ [0.5cm] 
                {\large\bfseries\textsc{Tutora:}} \\ 
                {\large\textsc{#4}} \\ [0.5cm]
                {\large\bfseries #5} 
            \end{tabular}
        \end{flushright}
    \end{center}
    \clearpage
}

\begin{document}

\frontmatter
\pagenumbering{gobble}
\nuevaportada{Ciencia de Datos}{Problema de Localización de Centros k-Balanceado Multiobjetivo}{Manuel Rubio Martínez}{Anna Martínez Gavara}{Abril, 2025}

\clearpage
\thispagestyle{empty} \mbox{} \clearpage

\newpage
\tableofcontents

\chapter{Introducción}

\section{Antecedentes}

\section{Problema de localización}

\section{Motivación}

\section{Aplicaciones}


\chapter{Preliminares}
\section{Conceptos básicos de la teoría de grafos}

\begin{defi}
Un \textbf{grafo} $G$ es una estructura formada por un par $G=(V,E)$ donde $V$ es el conjunto de vertices, el cual
tiene que ser no vacío y finito, y $E$ es un conjunto de pares no ordenados de elementos . A los elementos
de $V$ se les llama \textbf{vértices} o \textbf{nodos} y a los elementos de $E$ se les llama \textbf{aristas}.

\smallskip

Si una arista está compuesta por 2 elementos iguales, se dice que es un \textbf{lazo}.

\end{defi}
\bigskip

\chapter{Modelo Matemático}

\section{Definición}
%Funciones objetivo, restricciones, etc

\chapter{Grasp}




\chapter{Experimentación y Resultados}

% Instancias, comparativa, CASO DE ESTUDIO

\chapter{Conclusiones}





\begin{thebibliography}{X}
    \bibitem{Juegos} \textsc{Martín Novo, Eduardo; Méndez Alonso, Alfredo}.
    \textit{Aplicaciones de la teoría de grafos a algunos juegos de estrategia.}
    \url{https://redined.educacion.gob.es/xmlui/handle/11162/13874}

    \bibitem{Caníbales} Caníbales y misioneros: \url{https://culturacientifica.com/2016/05/04/problema-los-tres-caballeros-los-tres-criados/}

    \bibitem{Laberintos} Laberintos: \url{https://www.cantab.net/users/michael.behrend/repubs/maze_maths/pages/tarry_lab_en.html}
    
    \bibitem{Ajedrez} Sobre ajedrez: \url{https://chess24.com/es/informate/noticias/como-piensan-y-operan-los-modulos-de-analisis}

    \bibitem{Mantel} Teorema de Mantel: \url{https://lidicky.name/oldteaching/20.569X/l13%20-%20Mantel.pdf}

    \bibitem{Turan} Vascimini, Vincent. (2017). Extremal Graph Theory: Turán’s Theorem. 
    In BSU Honors Program Theses and Projects. Item 234. Available at: \url{https://vc.bridgew.edu/honors_proj/234}

    \bibitem{otros_teoremas} \textsc{Widdershoven, Cas.} \textit{Extremal graphs and the Erd{\H{o}}s-Stone-Simonovits theorems.}
    \url{https://studenttheses.uu.nl/handle/20.500.12932/23972}

\end{thebibliography}


\end{document}

